\documentclass[11pt, twoside]{report}
\usepackage[utf8]{inputenc}
\usepackage[hmargin=2.5cm,vmargin=2.5cm,bmargin=2.5cm]{geometry}
\usepackage{graphicx}  		% display images
\usepackage{rotating}
\usepackage{natbib}
\usepackage{tikz} 
\usepackage[portuguese]{babel}
\usepackage{indentfirst}
\usepackage[colorlinks=true,linkcolor=black,urlcolor=black,bookmarksopen=true]{hyperref} % Make hyperlinks in index
\usepackage{bookmark} 		% Bookmarks for pdf file
\usepackage{float} 			% colocar as imegens e tabelas dentro do texto
\usepackage{fancyhdr}
\pagestyle{fancy}
\usepackage{tabularx} 		% x column in table can jump a line
\usepackage{setspace} 		% espçamento entre linhas
\usepackage{fancyhdr} 		% creates fancy footers and headers
\raggedbottom				% makes bottom of page more empy to make sure previous text doesnt have vertical gaps
\usepackage{ltxtable}
\pagestyle{fancy}
\lfoot{{\footnotesize Tecnologias de Informação}}
\rfoot{{\footnotesize ESTGA}}
\rhead{PTDW}
\lfoot{Calendário de exames}

\renewcommand{\footrulewidth}{1pt}%criar uma linha no que separa o rodapé
%\usepackage{appendix}
%\noindent %sem indentação
%\newcommand{\annexname}{Anexo}
%\makeatletter % treat @ as a letter instead of a control word.
%
%\newcommand\annex{\par
%	\setcounter{chapter}{0}
%	\setcounter{section}{0}
%	\renewcommand\appendixname{Anexo}
%	\renewcommand\appendixpagename{Anexos}
%	\renewcommand{\appendixtocname}{Anexos}
%	\gdef\@chapapp{\annexname}
%	\gdef\thechapter{\@Roman\c@chapter}
%	\renewcommand{\theHchapter}{\annexname.\thechapter}
%	\addappheadtotoc
%}\makeatother


\begin{document}
	
\onehalfspacing % espaçamento de 1,5 entre linhas

	
%	\lhead{Escola Superior de Tecnologia e Gestão de Águeda\\
%		Licenciatura em Tecnologias da Informação
%	}
	
	%\rhead{Capítulo \thechapter}
	\pagenumbering{roman}
	
	\begin{titlepage}
		\centering
		\scshape\Huge Calendário Exames\par
		\vspace{0.9cm}
		
		\scshape\large Projeto Temático em Desenvolvimento Web \\
		\vspace{0.3cm}
		\scshape\large 1º semestre de 2021/2022\par
		\vspace{0.4cm}
		\centering
		%\includegraphics[width=10cm]{}\par
		
		\vspace{1cm}
		
		\large
		Autores\\
		Gonçalo Tavares, Nº 92382  \\
		Bruno Lopes, Nº 86217 \\
		Leonardo Silva, Nº 95381 \\
		Ricardo Fernandes, Nº 49880 \\
		Sofia Rocha, Nº 99991 \\
		
		\vspace{1cm}
		
		\centering
		\includegraphics[width=10cm]{image/AssB_vertical_cor.png}
		
		\newpage
		\thispagestyle{plain}%retira cabeçalho e rodape
		\thispagestyle{empty}%retira a numeração da pagina
		\centering
		\scshape\Huge Calendário Exames \par
		\vspace{1cm}
		
		\scshape\large Projeto Temático em Desenvolvimento Web\par
		\vspace{1cm}
		\scshape\large 1º semestre de 2021/2022\par
		\vspace{4cm}
		
		
		
		\large
		Autores\\
		Bruno Lopes, Nº 86217 \\
		Gonçalo Tavares, Nº 92382  \\
		Leonardo Silva, Nº 95381 \\
		Ricardo Fernandes, Nº 49880  \\
		Sofia Rocha, Nº 99991 \\
		
		\vspace{1cm}
		Orientadores\\
		Rita Santos \\
		Fábio Marques\\
		\vspace{4cm}
		
		\centering
		\includegraphics[width=10cm]{image/AssB_vertical_cor}
		
	\end{titlepage}

	\newpage
	\setcounter{page}{1} % começa a contar a paginas no numero 1
	\tableofcontents % Índice de conteúdos
	\thispagestyle{plain} % retira cabeçalho e rodape
	\thispagestyle{empty} % retira a numeração da pagina
	\newpage
	\listoftables % Lista de tabelas
	\newpage
	\listoffigures % Lista de figuras
	
	\newpage
	\pagenumbering{arabic}
	
	\chapter{Introdução}
	\section{Objetivos da aplicação}
	
	
	\chapter{Estado de arte}
	\chapter{Planificação do projeto}

 	

	\chapter{Análises dos utilizadores}
	
	\chapter{Modelo de requisitos}
	\section{Requisitos funcionais}

	\pagebreak % table is too long, this will stay here temorarily until solution is foundj
	\begin{center}
				\begin{tabularx}{\textwidth}{|c|X|c|}
				\hline
				\textbf{Refª }	& \textbf{Descrição do requisito} & \textbf{Prioridade} \\
				\hline
				RF1. & Criação de calendários com slots para colocação de exames & Alta \\
				\hline
				RF2. & As disciplinas e cursos podem ser inseridos manualmente & Alta\\
				\hline
				RF3. & Importar ficheiros com a configuração de salas, cadeiras e docentes em formato .csv & Alta \\
				\hline
				RF4. &  Restrição de marcação de exames com aviso a cor no caso de incongruência da informação & Alta \\
				\hline
				RF5. & Os exames podem ser marcados em três turnos: manhã (às 9h30), tarde (às 14h) e noite (às 18h30) por padrão & Alta \\
				\hline
				RF6. & O utilizador pode criar épocas de avaliação adicionando um nome e uma data de início e fim & Alta \\
				\hline
				RF7. & A criação de um novo calendário deverá sempre partir do início sem nenhuma configuração associada & Alta\\
				\hline
				RF8. & Permitir retroceder nas alterações feitas & Baixa \\
				\hline
				RF9. & Guardar e visualizar calendários de exames de anos anteriores sem informações específicas  & Média \\
				\hline
				RF10.  & Funcionalidade de pesquisar por cadeiras com filtro por cursos, ano, semestre e época de avaliação  & Alta \\
				\hline
				RF11. & O calendário deverá omitir (????)os domingos e feriados & Alta \\
				\hline
				RF12. & Configurar tipo de sala com equipamento, lotação total e máxima & Alta \\
				\hline
				RF13. & Associar manualmente vigilantes por sala evitando conflitos presenciais & Alta \\
				\hline
				RF14. & Permitir associar na ficha do docente dias em que os mesmos não estão disponíveis & Alta\\
				\hline
				RF15. & Permitir colocar restrições arbitrárias introduzidas pelo utilizador & Baixa \\
				\hline
				RF16. & Exportação de calendários em formato .pdf & Alta \\
				\hline
				RF17. & Permitir a opção de exportar o calendário em língua Inglesa & Baixa \\
				\hline
				RF18. & Permitir inserir docentes &Alta\\
				\hline
				RF19. & Permitir editar manualmente docentes & Alta\\
				\hline
				RF.20 &O utilizador pode associar a cada exame vários vigilantes&Média\\
				\hline
				RF.22 &	O utilizador pode associar mais do que uma sala a um exame&Alta\\
				\hline
			\end{tabularx}
	\end{center}

	
	\section{Requisitos não funcionais}
	\subsection{Requisitos de interface e facilidade de uso}
	PERGUNTAR SOBRE CURSOS COM O MESMO EXAME
	
	Colocar informação sobre os avisos
	
		\begin{center}
		\begin{tabularx}{\textwidth}{|c|X|c|}
			\hline
			\textbf{Refª }	& \textbf{Descrição do requisito} & \textbf{Prioridade} \\
			\hline
			RIF1 & As disciplinas e cursos podem ser inseridas através de \textit{drag e drop} &Alta\\
			\hline
			RIF2 & Interface responsiva permitindo a sua visualização em ambiente mobile &Alta\\
			\hline
			RIF3 & Linguagem padrão em Português de Portugal &Alta\\
			\hline
			RIF4 &Mostrar um aviso de alta prioridade se houver sobreposições de exames &Alta\\
			\hline
			RIF5 &Mostrar um aviso de alta prioridade o docente não estiver disponível&Alta\\
			\hline
			RIF6&Mostrar um aviso de alta prioridade se a sala não estiver disponível&Alta\\
			\hline
			RIF7&Mostrar um aviso de alta prioridade se o curso for diurno e colocar um exame no turno da noite e vice-versa & Alta\\
			\hline
			RIF8&Mostrar um aviso de alta prioridade se os docentes associados ao exame forem duplicados& Alta\\
			\hline
			RIF9&Mostrar um aviso de alta prioridade se o exame necessitar de uma sala de informática e não for associada sala desse tipo&Alta\\
			\hline
			RIF10&Mostrar um aviso de alta prioridade se houver mais alunos inscritos do que  lotação máxima da sala&Alta\\
			\hline
			RIF11&Mostrar um aviso de média prioridade se houver exames marcados no mesmo dia e hora do mesmo curso mas anos diferentes&Média\\
			\hline
			RIF12&&\\
			\hline
			RIF13&&\\
			\hline
			RIF14&&\\
			\hline
			RIF15&&\\
			\hline
			RIF16&&\\
			\hline
			RIF17&&\\
			\hline
		\end{tabularx}
		\label{requisitosdeinterface}
	\end{center}
	\subsection{Requisitos de segurança e integridade dos dados}
	
	
	\begin{center}	
	\begin{tabularx}{\textwidth}{|c|X|c|}
		\hline
		\textbf{Refª }	& \textbf{Descrição do requisito} & \textbf{Prioridade} \\
		\hline
		RSI1 &Não pode haver múltiplas associações dos docentes no mesmo turno &Alta\\
		\hline
		RSI2 &O histórico não pode ter associações a outras tabelas da base de dados  &Alta\\
		\hline
		RSI3 & Uma conta de utilizador (????????????????????)&\\
		\hline
		RSI4 & &\\
		\hline
	\end{tabularx}
	\label{requisitosdeseguranca}
\end{center}
	
	\subsection{Requisitos de interface com sistemas externos e ambientes de execução}
		\begin{center}[
		\begin{tabularx}{\textwidth}{|c|X|c|}
			\hline
			Rfª & Nome & Prioridade\\
			\hline
			RSA1 & Suportar no Chrome (versão 90 ou superior) &Alta\\
			\hline
			RSA2 &Suportar no Firefox (versão 87 ou superior) &Alta\\
			\hline
			RSA3 &Suportar no Microsoft Edge (versão 89 ou superior)&Alta\\
			\hline
			RSA4 &Suportar no Safari (versão 14 ou superior)&Alta\\
			\hline
			RSA5 &Ter acesso à internet&Alta\\
			\hline
		\end{tabularx}
		\label{requisitosdesistemas}
	\end{center}
	
	\chapter{Modelo de casos de utilização}
	\section{Diagrama de casos de utilização}
	\section{Seleção dos casos de utilização}
	\section{Descrição dos casos de utilização}
	
	\chapter{Prototipagem}
	\section{Protótipo de baixa fidelidade}
	\subsection{Wireframes}
	\subsection{Diagrama de user flow}
	\subsection{Testes}
	\subsubsection{Análise de resultados}
	
	\section{Protótipo de alta fidelidade}
	\subsection{Desenvolvimento do protótipo}
	\subsection{Guia de estilos}
	\subsection{Testes}
	\subsubsection{Análise de resultados}
	
	\chapter{Implementação do modelo de dados presistentes}
	\section{Estrutura da base de dados}
	\subsection{Base de dados - factories}
	\section{Arquitetura do sistema - Modelo MVC}
	\subsection{Models e Controllers}
	
	\chapter{Primeira versão da aplicação}
	\section{Implementação de funcionalidades}
	
	\chapter{Testes finais}
	\section{Testes com potenciais clientes}
	\section{Testes de acessibilidade}
	\section{Análise de resultados}
	
	\chapter{Lançamento da versão final}
	\section{Alocação da aplicação no servidor}
	
	\pagebreak
	wasd
	wasd
	
	
	wasd
	
	
	\pagebreak
	
	asd
	
	wasd
	\chapter{Reflexão crítica e conclusão}
	
	

	\bibliographystyle{ieeetr}
	\bibliography{}
	
\end{document}	
	
	
	
	
