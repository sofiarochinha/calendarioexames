\documentclass[11pt, twoside]{report}
\usepackage[utf8]{inputenc}
\usepackage[hmargin=2.5cm,vmargin=2.5cm,bmargin=2.5cm]{geometry}
\usepackage{graphicx}  		% display images
\usepackage{rotating}
\usepackage{natbib}
\usepackage{tikz} 
\usepackage[portuguese]{babel}
\usepackage{indentfirst}
\usepackage[colorlinks=true,linkcolor=black,urlcolor=black,bookmarksopen=true]{hyperref} % Make hyperlinks in index
\usepackage{bookmark} 		% Bookmarks for pdf file
\usepackage{float} 			% colocar as imegens e tabelas dentro do texto
\usepackage{fancyhdr}
\pagestyle{fancy}
\usepackage{tabularx} 		% x column in table can jump a line
\usepackage{setspace} 		% espçamento entre linhas
\usepackage{fancyhdr} 		% creates fancy footers and headers
\raggedbottom				% makes bottom of page more empy to make sure previous text doesnt have vertical gaps
\usepackage{makecell} %forçar linhas dentro de tabelas
\pagestyle{fancy}
\lfoot{{\footnotesize Tecnologias de Informação}}
\rfoot{{\footnotesize ESTGA}}
\rhead{Projeto Temático em Desenvolvimento Web}
\lfoot{Calendário de exames}

%\newcommand{\version}{\input{"git latexdiff ~HEAD"}} tentei

\renewcommand{\footrulewidth}{1pt}%criar uma linha no que separa o rodapé
%\usepackage{appendix}
%\noindent %sem indentação
%\newcommand{\annexname}{Anexo}
%\makeatletter % treat @ as a letter instead of a control word.
%
%\newcommand\annex{\par
%	\setcounter{chapter}{0}
%	\setcounter{section}{0}
%	\renewcommand\appendixname{Anexo}
%	\renewcommand\appendixpagename{Anexos}
%	\renewcommand{\appendixtocname}{Anexos}
%	\gdef\@chapapp{\annexname}
%	\gdef\thechapter{\@Roman\c@chapter}
%	\renewcommand{\theHchapter}{\annexname.\thechapter}
%	\addappheadtotoc
%}\makeatother


\begin{document}
\onehalfspacing%espaçamento de 1,5 entre linhas

	
%	\lhead{Escola Superior de Tecnologia e Gestão de Águeda\\
%		Licenciatura em Tecnologias da Informação
%	}
	
	%\rhead{Capítulo \thechapter}
	\pagenumbering{roman}
	
	\begin{titlepage}
		\centering
		\scshape\Huge Calendário Exames\par
		\vspace{0.9cm}
		
		\scshape\large Projeto Temático em Desenvolvimento Web \\
		\vspace{0.3cm}
		\scshape\large 1º semestre de 2021/2022\par
		\vspace{0.4cm}
		\centering
		%\includegraphics[width=10cm]{}\par
		
		\vspace{1cm}
		
		\large
		Autores\\
		Gonçalo Tavares, Nº 92382  \\
		Bruno Lopes, Nº 86217 \\
		Leonardo Silva, Nº 95381 \\
		Ricardo Fernandes, Nº 49880 \\
		Sofia Rocha, Nº 99991 \\
		
		\vspace{1cm}
		
		\centering
		\includegraphics[width=10cm]{image/AssB_vertical_cor.png}
		
		\newpage
		\thispagestyle{plain}%retira cabeçalho e rodape
		\thispagestyle{empty}%retira a numeração da pagina
		\centering
		\scshape\Huge Calendário Exames \par
		\vspace{1cm}
		
		\scshape\large Projeto Temático em Desenvolvimento Web\par
		\vspace{1cm}
		\scshape\large 1º semestre de 2021/2022\par
		\vspace{4cm}
		
		
		
		\large
		Autores\\
		Bruno Lopes, Nº 86217 \\
		Gonçalo Tavares, Nº 92382  \\
		Leonardo Silva, Nº 95381 \\
		Ricardo Fernandes, Nº 49880  \\
		Sofia Rocha, Nº 99991 \\
		
		\vspace{1cm}
		Orientadores\\
		Rita Santos \\
		Fábio Marques\\
		\vspace{4cm}
		
		\centering
		\includegraphics[width=10cm]{image/AssB_vertical_cor}
		
	\end{titlepage}

	\newpage
	\setcounter{page}{1} % começa a contar a paginas no numero 1
	\tableofcontents % Índice de conteúdos
	\thispagestyle{plain} % retira cabeçalho e rodape
	\thispagestyle{empty} % retira a numeração da pagina
	\newpage
	\listoftables % Lista de tabelas
	\newpage
	\listoffigures % Lista de figuras
	
	\newpage
	\pagenumbering{arabic}
	
	\chapter{Introdução}
	\section{Objetivos da aplicação}
	
	
	\chapter{Estado de arte}
	\chapter{Planificação do projeto}

 	

	\chapter{Análises dos utilizadores}
	
	\chapter{Modelo de requisitos}
	
	
	\section{Requisitos funcionais}
	
	\begin{table}[H]
		\centering
		\caption{Requisitos funcionais}	
		\vspace{0.5cm}
		\begin{tabular}{|l|l|l|}
			\hline
			Rfª & Nome & Prioridade\\
			\hline
			RF1 & As disciplinas e cursos podem ser inseridas manualmente& alta\\
			\hline
			RF2 & Deve ter a possibilidade de exportar toda a informação em pdf&alta\\
			\hline
			RF3 &\makecell[l]{O utilizador pode importar as informações sobre as salas, docentes, disciplinas e\\ cursos apartir de um ficheiro csv}&alta\\
			\hline
			RF4 &Mostrar avisos em caso de incongruência da informação&\\
			\hline
			RF5 &\makecell[l]{Mostrar um aviso se houver sobreposição de exames (que sejam na mesma hora e \\na mesma sala)}&alta\\
			\hline
			RF6& \makecell[l]{Mostrar um aviso se o professor associado ao exame não tiver disponível \\(caso já esteje ocupado com outro exame ou trabalhar nesse dia e nessa hora)}&alta\\
			\hline
			RF7& \makecell[l]{Os exames podem ser marcados em três turnos: manhã (às 9h30), tarde (às 14h) e \\noite (às 18h30) por padrão}& alta\\
			\hline
			RF8& \makecell[l]{O utilizador pode criar épocas de avaliação adicionando um nome e uma data de \\início e fim}& alta\\
			\hline
			RF9& O utilizador pode alterar qualquer informação sobre os exames, docentes ou os cursos&alta\\
			\hline
			RF10&\makecell[l]{Ao exportar para pdf se o curso for diurno não irá aparecer o turno noturno \\e vice-versa.}&baixa\\
			\hline
			RF11&\makecell[l]{A criação de um novo calendário deverá sempre partir do início sem \\nenhuma configuração associada}&alta\\
			\hline
			RF12&Deve permitir a visualização de calendários de avaliações criados anteriormente &alta\\
			\hline
			RF13&O utilizador pode pesquisar por disciplinas, curso, ano, semestre e época de avaliação&média\\
			\hline
			RF14&As salas devem ser identificadas como de informática, redes ou salas normais &alta\\
			\hline
			RF15&O utilizador pode escolher quais são os docentes que serão vigilantes nos exames&alta\\
			\hline
			RF16&Permitir retrodecer nas alterações feitas&baixa\\
			\hline
			RF17&O utilizador pode associar qualquer disciplina do curso escolhido ao calendário&alta\\
			\hline
		\end{tabular}
		\label{requisitosfuncionais}
	\end{table}
	
	
	\section{Requisitos não funcionais}
	
	
	
	\subsection{Requisitos de interface e facilidade de uso}
	\begin{table}[H]
		\centering
		\caption{Requisitos de interface e facilidade de uso}	
		\vspace{0.5cm}
		\begin{tabular}{|l|l|l|}
			\hline
			Rfª & Nome & Prioridade\\
			\hline
			RIF1 & As disciplinas e cursos podem ser inseridas através de \textit{drag e drop} &alta\\
			\hline
			RIF2 & &\\
			\hline
			RIF3 &&\\
			\hline
			RIF4 &&\\
			\hline
			RIF5 &&\\
			\hline
			RIF6&&\\
			\hline
			RIF7& & \\
			\hline
			RIF8& & \\
			\hline
			RIF9& &\\
			\hline
			RIF10&&\\
			\hline
			RIF11&&\\
			\hline
			RIF12& &\\
			\hline
			RIF13&&\\
			\hline
			RIF14&&\\
			\hline
			RIF15&&\\
			\hline
			RIF16&&\\
			\hline
			RIF17&&\\
			\hline
		\end{tabular}
		\label{requisitosdeinterface}
	\end{table}
	\subsection{Requisitos de segurança e integridade dos dados}
	\subsection{Requisitos de interface com sistemas externos e ambientes de execução}
	

	\begin{table}[H]
		\centering
		\caption{Requisitos de interface com sistemas externos e ambientes de execução}	
		\vspace{0.5cm}
		\begin{tabular}{|l|l|l|}
			\hline
			Rfª & Nome & Prioridade\\
			\hline
			RSA1 & Suportar no Chrome (versão 90 ou superior) &alta\\
			\hline
			RSA2 &Suportar no Firefox (versão 87 ou superior) &\\
			\hline
			RSA3 &Suportar no Microsoft Edge (versão 89 ou superior)&\\
			\hline
			RSA4 &Suportar no Safari (versão 14 ou superior)&\\
			\hline
			RSA5 &Ter acesso à internet&\\
			\hline
		\end{tabular}
		\label{requisitosdesistemas}
	\end{table}
	
	\chapter{Modelo de casos de utilização}
	\section{Diagrama de casos de utilização}
	\section{Seleção dos casos de utilização}
	\section{Descrição dos casos de utilização}
	
	\chapter{Prototipagem}
	\section{Protótipo de baixa fidelidade}
	\subsection{Wireframes}
	\subsection{Diagrama de user flow}
	\subsection{Testes}
	\subsubsection{Análise de resultados}
	
	\section{Protótipo de alta fidelidade}
	\subsection{Desenvolvimento do protótipo}
	\subsection{Guia de estilos}
	\subsection{Testes}
	\subsubsection{Análise de resultados}
	
	\chapter{Implementação do modelo de dados presistentes}
	\section{Estrutura da base de dados}
	\subsection{Base de dados - factories}
	\section{Arquitetura do sistema - Modelo MVC}
	\subsection{Models e Controllers}
	
	\chapter{Primeira versão da aplicação}
	\section{Implementação de funcionalidades}
	
	\chapter{Testes finais}
	\section{Testes com potenciais clientes}
	\section{Testes de acessibilidade}
	\section{Análise de resultados}
	
	\chapter{Lançamento da versão final}
	\section{Alocação da aplicação no servidor}
	
	\pagebreak
	wasd
	wasd
	
	
	wasd
	
	
	\pagebreak
	
	asd
	
	wasd

	\chapter{Reflexão crítica e conclusão}
	
	

	\bibliographystyle{ieeetr}
	\bibliography{}
	
\end{document}	
	
	
	
	
