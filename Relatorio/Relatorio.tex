\documentclass[11pt, twoside]{report}
\usepackage[utf8]{inputenc}

\usepackage[hmargin=2.5cm,vmargin=2.5cm,bmargin=2.5cm]{geometry}
\usepackage{rotating}
\usepackage{natbib}
\usepackage{tikz} 
\usepackage[portuguese]{babel}
\usepackage{indentfirst}
\usepackage{float}%colocar as imegens e tabelas dentro do texto
\usepackage{fancyhdr}
\pagestyle{fancy}
\usepackage{setspace}%espçamento entre linhas
\renewcommand{\footrulewidth}{1pt}%criar uma linha no que separa o rodapé
\usepackage{appendix}
%\noindent sem indentação
\newcommand{\annexname}{Anexo}
\makeatletter % treat @ as a letter instead of a control word.

\newcommand\annex{\par
	\setcounter{chapter}{0}
	\setcounter{section}{0}
	\renewcommand\appendixname{Anexo}
	\renewcommand\appendixpagename{Anexos}
	\renewcommand{\appendixtocname}{Anexos}
	\gdef\@chapapp{\annexname}
	\gdef\thechapter{\@Roman\c@chapter}
	\renewcommand{\theHchapter}{\annexname.\thechapter}
	\addappheadtotoc
}\makeatother


\begin{document}
	
\onehalfspacing%espaçamento de 1,5 entre linhas

	
	\lhead{Escola Superior de Tecnologia e Gestão de Águeda\\
		Licenciatura em Tecnologias da Informação
	}
	\lfoot{Calendário de exames}
	\rhead{Capítulo \thechapter}
	\pagenumbering{roman}
	
	\begin{titlepage}
		\centering
		\scshape\Huge Calendário Exames\par
		\vspace{0.9cm}
		
		\scshape\large Projeto Temático em Desenvolvimento Web \\
		\vspace{0.3cm}
		\scshape\large 1º semestre de 2021/2022\par
		\vspace{0.4cm}
		\centering
		%\includegraphics[width=10cm]{}\par
		
		\vspace{1cm}
		
		\large
		Autores\\
		Gonçalo Tavares, Nº   \\
		Bruno Lopes, Nº 86217 \\
		Leonardo Silva, Nº 95381 \\
		Ricardo Luiz, Nº  \\
		Sofia Rocha, Nº 99991 \\
		
		\vspace{1cm}
		
		\centering
		\includegraphics[width=6cm]{logoestga}
		
		\newpage
		\thispagestyle{plain}%retira cabeçalho e rodape
		\thispagestyle{empty}%retira a numeração da pagina
		\centering
		\scshape\Huge Calendário Exames \par
		\vspace{1cm}
		
		\scshape\large Projeto Temático em Desenvolvimento Web\par
		\vspace{1cm}
		\scshape\large 1º semestre de 2021/2022\par
		\vspace{4cm}
		
		
		
		\large
		Autores\\
		Gonçalo Tavares, Nº   \\
		Bruno Lopes, Nº 86217 \\
		Leonardo Silva, Nº 95381 \\
		Ricardo Luiz, Nº  \\
		Sofia Rocha, Nº 99991 \\
		
		\vspace{1cm}
		Orientadores\\
		Rita Santos \\
		Fábio Marques\\
		\vspace{4cm}
		
		\centering
		\includegraphics[width=6cm]{logoestga}
		
	\end{titlepage}

	\newpage
	\setcounter{page}{1}%começa a contar a paginas no numero 1
	\tableofcontents %Índice de conteúdos
	\thispagestyle{plain}%retira cabeçalho e rodape
	\thispagestyle{empty}%retira a numeração da pagina
	\newpage
	\listoftables %Lista de tabelas
	\newpage
	\listoffigures %Lista de figuras
	
	
	
	
	\newpage
	\pagenumbering{arabic}
	
	\chapter{Introdução}
	\section{Objetivos da aplicação}
	
	
	\chapter{Estado de arte}
	\chapter{Planificação do projeto}

 	

	\chapter{Análises dos utilizadores}
	
	\chapter{Modelo de requisitos}
	\section{Requisitos funcionais}
	\section{Requisitos não funcionais}
	\subsection{Requisitos de interface e facilidade de uso}
	\subsection{Requisitos de segurança e integridade dos dados}
	\subsection{Requisitos de interface com sistemas externos e ambientes de execução}
	
	\chapter{Modelo de casos de utilização}
	\section{Diagrama de casos de utilização}
	\section{Seleção dos casos de utilização}
	\section{Descrição dos casos de utilização}
	
	\chapter{Prototipagem}
	
	\chapter{Aplicação}
	\section{Guia de estilos}
	\section{Implementação da camada de apresentação}
	\section{Estrutura da base de dados}
	\subsection{Base de dados - factories}
	\section{Arquitetura do sistema - Modelo MVC}
	\subsection{Models e Controllers}
	\section{Implementação de funcionalidades}
	\chapter{Análise dos resultados}
	\chapter{Reflexão crítica e conclusão}
	
	

	\bibliographystyle{plain}
	\bibliography{}
	
	
\end{document}	
	
	
	
	
