\documentclass[12pt, twoside]{report}
\usepackage[utf8]{inputenc}
\usepackage[margin=2.5cm]{geometry}
\usepackage{graphicx}  		% display images
\usepackage{rotating}
\usepackage{tikz} 
\usepackage[portuguese]{babel}

\usepackage{mathptmx}
\usepackage{setspace}

\usepackage[colorlinks=true,linkcolor=black,urlcolor=black,bookmarksopen=true]{hyperref} % Make hyperlinks in index
\usepackage{bookmark} 		% Bookmarks for pdf file
\usepackage{float} 			% colocar as imegens e tabelas dentro do texto
\usepackage{fancyhdr}
\pagestyle{fancy}
\usepackage{tabularx} 		% x column in table can jump a line
\usepackage{setspace} 		% espçamento entre linhas
\usepackage{fancyhdr} 		% creates fancy footers and headers
\raggedbottom				% makes bottom of page more empy to make sure previous text doesnt have vertical gaps
\usepackage{ltablex}
\usepackage{pdflscape}
\usepackage{multirow}
\usepackage{minted} %colocação de códigos
\pagestyle{fancy}
\lfoot{{\footnotesize Tecnologias de Informação}}
\rfoot{{\footnotesize ESTGA}}
\rhead{PTDW}
\lfoot{Calendário de exames}
\usepackage{nomencl}
\makenomenclature
\renewcommand{\nomname}{Nomenclatura}%mudar o nome da secção

\renewcommand{\footrulewidth}{1pt}%criar uma linha no que separa o rodapé

\renewcommand\listoflistingscaption{Índices dos comandos e configurações} %lista dos codigos



\begin{document}
	
\onehalfspacing % espaçamento de 1,5 entre linhas

	\pagenumbering{roman}
	
	\begin{titlepage}
		\centering
		\scshape\Huge Calendário Exames\par
		\vspace{0.9cm}
		
		\scshape\large Projeto em Tecnologias da Informação \\
		\vspace{0.3cm}
		\scshape\large 1º semestre de 2021/2022\par
		\vspace{0.4cm}
		\centering
		
		\vspace{3cm}
		
		\large
		Autor\\
		Sofia Rocha, Nº 99991 \\
		
		\vspace{2cm}
		\large
		Águeda, mês, 2022 \\
		
		\vspace{4cm}
		
		\centering
		\includegraphics[width=10cm]{image/AssB_vertical_cor.png}
		
		
		\newpage
		\thispagestyle{plain}%retira cabeçalho e rodape
		\thispagestyle{empty}%retira a numeração da pagina
		\centering
		\scshape\Huge Calendário Exames \par
		\vspace{1cm}
		
		\scshape\large Projeto em Tecnologias da Informação\par
		\vspace{1cm}
		\scshape\large 2º semestre de 2021/2022\par
		\vspace{4cm}
		
		\large
		Autor\\
		Sofia Rocha, Nº 99991 \\
		
		\vspace{1cm}
		Orientador\\
		Fábio Marques\\
		
		\vspace{3cm}
		\large
		Águeda, mês, 2022 \\
		
		\vspace{2cm}
		\centering
		\includegraphics[width=10cm]{image/AssB_vertical_cor}
		
	\end{titlepage}

	\begin{abstract}
	\setstretch{1.5}
	\setlength{\parskip}{14pt}
	\setlength{\parindent}{0pt}
	
	
	Resumo máximo 300 palavras
	palavras-chave (6 no máximo)
	\end{abstract}
	
	\begin{abstract}
		Resumo em inglês máximo 300 palavras
		palavras-chave (6 no máximo)
	\end{abstract}

	\newpage
	\setcounter{page}{1} % começa a contar a paginas no numero 1
	\tableofcontents % Índice de conteúdos
	\thispagestyle{plain} % retira cabeçalho e rodape
	\thispagestyle{empty} % retira a numeração da pagina
	\newpage
	\listoftables % Lista de tabelas
	\newpage
	\listoffigures % Lista de figuras
	
	
	\newpage
	\pagenumbering{arabic}
	

	\chapter{Introdução}
	\setstretch{1.5}
	\setlength{\parskip}{14pt}
	\setlength{\parindent}{0pt}
Apresentação do tema do estágio/projeto e da EA;
Apresentação dos objetivos gerais e específicos do estágio/projeto;
Indicação da metodologia de trabalho adotada
Descrição da estrutura interna do relatório

	
	\chapter{Planificação do projeto}
	Identificação e descrição das atividades previstas para a realização
	do projeto (Plano de Trabalho)
	\subsection{Revisão do estado do projeto}
	\subsection{Planeamento do projeto}
	
	\chapter{Modelo de requisitos}
	\label{requisitos}
	\section{Requisitos funcionais}
	
	Como este projeto é uma continuação de um projeto anterior então na tabela \ref{requisitiosfuncionais} não se encontram todos os requisitos funcionais, mas sim aqueles que serão implementados. 
	
	
\def\arraystretch{1.5}
	\begin{center}
		\label{requisitiosfuncionais}
		\begin{longtable}{|m{1cm}|m{2.2cm}|m{10cm}|m{2cm}|}
			\caption{Requisitos funcionais a serem implementados}\\
			
			\hline			
			\textbf{Refª }	& \textbf{Categoria}&\textbf{Descrição do requisito} & \textbf{Prioridade} \\
			\hline
			
			
			RF.1 &Importação& Importação de ficheiros com a configuração de salas, disciplinas e docentes em formato \textbf{csv} com dois ou mais docentes& Alta \\
			\hline
			
			RF.2 &\multirow{2}{2cm}{Exportação}& Exportação de calendários em formato \textbf{pdf} & Alta \\
			
			RF.3 && Exportação o calendário em língua inglesa & Baixa \\
			\hline
			
			RF.4 &\multirow{2}{2cm}{Marcação de exames}& Associação de um ou mais vigilantes a cada exame & Alta\\
						
			RF.5 &&	Associação de uma ou mais salas a cada exame & Alta\\
			
			RF.6 &&Indicação do número de salas associadas & Alta\\
			\hline
		
			RF.9 &\multirow{6}{2cm}{Dados Auxiliares}& Inserção da lotação máxima das salas& Média \\
			
			RF.10 && Alterar a disponibilidade dos docentes & Alta\\
			
			RF.10 && Adicionar épocas & Alta\\
			
			RF.11 && Editar épocas & Média\\
			
			RF.12 && Eliminar épocas & Alta\\
			
			RF.13 && Inserir disponibilidade dos docentes & Alta\\
			\hline
			
			RF.16 &\multirow{9}{2cm}{Avisos}& Mostrar um aviso de alta prioridade se houver sobreposições de exames & Baixa\\
						
			RF.17 && Mostrar um aviso de alta prioridade se o docente não estiver disponível & Média \\
			
			RF.18 && Mostrar um aviso de alta prioridade se a sala não estiver disponível & Média\\
			
			RF.19 && Mostrar um aviso de alta prioridade se o curso for diurno e colocar um exame no turno da noite e vice-versa & Baixa\\
			
			RF.20&&Mostrar um aviso de alta prioridade se o docente associado ao mesmo exame for repetido & Alta \\
			
			RF.21 && Mostrar um aviso de alta prioridade se o exame necessitar de uma sala de informática e não for associada sala desse tipo & Alta\\
			
			RF.22 && Mostrar um aviso de alta prioridade se houver mais alunos inscritos do que  lotação máxima da sala & Baixa\\
			
			RF.23 && Mostrar um aviso de média prioridade se houver exames marcados no mesmo dia e hora do mesmo curso mas anos diferentes & Média\\
			
			RF.24 && Mostrar um aviso de média prioridade se o utilizador tentar exportar um calendário sem todos os exames marcados & Média\\
			
			\hline
			
			RF.25 &Autenticação& O utilizador só pode aceder à aplicação após a autenticação & Alta\\
			\hline
			
			RF.26 &\multirow{1}{2cm}{Criação de calendários}& Criação de épocas de avaliação adicionando um nome e uma data de início e fim & Alta \\
	
			\hline
			RF.30 &\multirow{2}{*}{Histórico}& Guardar e visualizar calendários de exames de anos anteriores (histórico)& Média \\
			
			RF.31 && Filtrar o histórico por curso, ano letivo, ano do curso, semestre e época& Média \\
			\hline
		\end{longtable}
	\end{center}



	
	\section{Requisitos não funcionais}
	
	Os requisitos não funcionais estão divididos em três categorias: requisitos de interface e facilidade de uso que representam todos os requisitos que melhorem a usabilidade da aplicação; requisitos de segurança e integridade dos dados e requisitos de interface com sistemas externos e ambientes de execução.
	
	\subsection{Requisitos de interface e facilidade de uso}

Assim que o utilizador inicia a sessão na aplicação este tem de facilmente entender como a aplicação está organizada.
Isto permite que o utilizador utilize a aplicação durante mais tempo e gerir os calendários de avaliações não seja frustrante.
	
	\begin{table}[H]
	\caption{Requisitos de interface e facilidade de uso}
	
	\begin{center}
		\begin{tabularx}{\textwidth}{|c|X|c|}
			\hline
			\textbf{Refª }	& \textbf{Descrição do requisito} & \textbf{Prioridade} \\
			\hline
			RIF1 & As disciplinas e cursos podem ser inseridas através de \textit{drag e drop} &Alta\\
			\hline
			RIF2 & Interface responsivo permitindo a sua visualização em ambiente mobile &Alta\\
			\hline
			RIF3 & Linguagem padrão em Português de Portugal &Alta\\
			\hline
			RIF4 & Há dois tipos de avisos distinguidos com texto e cor &Alta\\
			\hline
		\end{tabularx}
		\label{requisitosdeinterface}
	\end{center}
	\end{table}

	\subsection{Requisitos de segurança e integridade dos dados}
	
	Para que o utilizador possa colocar os seus dados na aplicação sem conter risco de vazar para utilizadores indesejados foram criados alguns requisitos de segurança referido na	tabela \ref{requisitosdeseguranca}.
	
\begin{table}[H]	
	\caption{Requisitos de segurança e integridade dos dados}
	
	
	\begin{center}
		\begin{tabularx}{\textwidth}{|c|X|c|}
			\hline
			\textbf{Refª }	& \textbf{Descrição do requisito} & \textbf{Prioridade} \\
			\hline
			RSI1 &O histórico não pode ter associações a outras tabelas da base de dados  &Alta\\
			\hline
			RSI2 & Uma única conta de utilizador&Alta\\
			\hline
		\end{tabularx}
		\label{requisitosdeseguranca}
	\end{center}
\end{table}


	\subsection{Requisitos de interface com sistemas externos e ambientes de execução}
	
	A aplicação por ambiente da disciplina irá ser programada em linguagens web,
	consequentemente não é necessário definir que sistema operativos pode a aplicação ser executada.
	No entanto é crucial ter acesso à rede da Universidade de Aveiro e um dos navegadores definidos na tabela \ref{requisitosdesistemas}.
	
	\def\arraystretch{1.5}
	\begin{table}[H]
		\caption{Requisitos de interface com sistemas externos e ambientes de execução}
		\begin{center}
			\begin{tabularx}{\textwidth}{|c|X|c|}
				\hline
				\textbf{Refª }	& \textbf{Descrição do requisito} & \textbf{Prioridade}\\
				\hline
				RSA1 & Suportar Browsers com motor renderização webkit/blink (Chrome, Edge, Safari, Brave, etc.)  & Alta \\
				\hline
				RSA2 & Suportar Firefox ESR e outros derivados de gecko/quantum & Alta \\
				\hline
				RSA5 & Estar conectado à rede da Universidade de Aveiro & Alta\\
				\hline
			\end{tabularx}
			\label{requisitosdesistemas}
		\end{center}
	\end{table}
		
	
	\chapter{Modelo de casos de utilização}
	
	
	\section{Diagrama de casos de utilização}
	\label{diagrama}
	


	\section{Seleção dos casos de utilização}

	
	\section{Descrição dos casos de utilização}
	\label{descricaousecase}
	


	
	\chapter{Prototipagem}
	
	\section{Protótipo de baixa fidelidade não funcional}
	
	\subsection{Wireframes}

	
	\subsection{Diagrama de user flow}

	\subsection{Testes com potenciais utilizadores}

	\subsubsection{Análise de resultados}
	\section{Protótipo de alta fidelidade não funcional}
	
	\subsection{Desenvolvimento do protótipo}

	\subsection{Guia de estilos}
	
	
	\subsection{Testes com potenciais utilizadores}
	
	\subsubsection{Análise de resultados}
	
	\subsection{Testes de acessibilidade e desempenho}
	
	\chapter{Implementação do modelo de dados persistentes}
	\section{Estrutura da base de dados}
	
	\section{Arquitetura do sistema - modelo MCV}
	
	
	

	\chapter{Primeira versão da aplicação}
	\section{Implementação de funcionalidades}

		
	
	\subsection{Análise de resultados}
	
	
	
	\chapter{Análise de resultados}
	
	
	
	
	\chapter{Lançamento da versão final}
	
	\section{Alocação da aplicação no servidor}
	
	
	
	\chapter{Conclusão}
	
	Atividades desenvolvidas
	• Estratégias de trabalho adotadas
	• Tecnologias utilizadas
	• Planeamento previsto e o efetivamente executado
	• Sugestões que possam colmatar eventuais lacunas do
	trabalho realizado
	
	\bibliographystyle{ieeetr}
	\bibliography{citations}
	\pagenumbering{roman}
	\pagestyle{empty}
	
	
\end{document}	
	
	
	
	
